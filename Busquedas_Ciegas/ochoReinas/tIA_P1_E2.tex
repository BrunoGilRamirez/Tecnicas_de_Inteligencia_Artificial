\documentclass[11pt, letterpaper]{article}
\usepackage[utf8]{inputenc}
\usepackage[spanish]{babel}
\usepackage{graphicx}
\graphicspath{images/}
\title{Programa 1: El programa de las 8 reinas}
\author{Gil Ramírez Bruno\\Moguel Silva José Miguel\\Ramirez Santamaria Isaí\\Zacarias Daniel Luis Alberto}
\date{9 de Marzo de 2023}
\begin{document}
\maketitle
\pagebreak
\tableofcontents
\pagebreak
\section{Planteamiento del problema}
El problema de las ocho reinas consiste en poner ocho reinas en el tablero de ajedrez sin que se amenacen.
\begin{figure}[!h]
    \centering
    \includegraphics{images/Reina.png}
\end{figure}

\smallskip
Como cada reina puede amenazar a todas las reinas que estén en la misma
fila y misma diagonal, cada una ha de situarse en una fila y diagonal diferente.
\section{Metodología}
\subsection{Estados inicial y final}
\subsection{Función objetivo}
\subsection{Función sucesor}
\subsection{Orden de los operadores}
\section{Diseño y descripción del algoritmo}
\section{Reporte de la trayectoria}
\section{Análisis de complejidad}
\section{Apéndices}
\end{document}